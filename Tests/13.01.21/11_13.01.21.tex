\documentclass[12pt, a4paper, oneside]{article}	% {article|letter|report}

\newcommand{\studentPreambleFolder}{C:/Users/DGL/dotfiles/LaTeX}
\input{\studentPreambleFolder/preamble}

\fancyheadoffset[R]{0.05cm} %так можно регулировать ширину колонтитула
\pagestyle{fancy}
\fancyhead[L]{\small Подготовлено для 11 класса LancmanSchool, 13.01.21}
\fancyhead[R]{\small denisov0gleb@gmail.com}

\begin{document}

\begin{enumerate}[1.] % \marginpar{Д{\slash}З}
	\item
		Фамилия

		\hrulefill

	\item
		Имя

		\hrulefill

	\item
		Что нравится в химии?

		\hrulefill

		\hrulefill

	\item
		Что \textbf{не} нравится в химии?

		\hrulefill

		\hrulefill

	\item
		Какие есть хобби и интересы?

		\hrulefill

		\hrulefill

	\item
		Напишите названия следующих элементов:

		\ce{N, Ba, H, K, P}

		\hrulefill

		\hrulefill

	\item
		Назовите следующие соединения и укажите их классы:

		\ce{HNO3, SO2, Ca(OH)2}

		\hrulefill

		\hrulefill
	
		\hrulefill

	\item
		Определите степени окисления всех элементов в \ce{P2O3}:

		\hrulefill

		\hrulefill
		
	\item
		Вычислите молекулярную массу \ce{CO2}:

		\hrulefill

		\hrulefill

\end{enumerate}

\end{document}
